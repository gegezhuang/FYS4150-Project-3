

For a time-dependent electric field we repeated simulations of 100 $\text{Ca}^{2+}$-ions with amplitudes $f = 0.1, 0.4, 0.7$. We did the simulations for evenly spaced frequencies $\omega_V$ in the interval $(0.2, 2.5)\text{MHz}$, and counted the number of particles left in the penning trap after $500\mu s$. Figure (TODO) shows that for $f=0.1$ there is a significant lowering of the number of particles left in the trap when the frequency $\omega_V = 0.62$. For an amplitude of $f=0.4$, the figure shows that the number of particles left is 0 for the tested frequencies between $0.56$ and $0.66$, and that there is a lowering of particles left for $\omega_V=0.3$. Similarly for $f=0.7$, the number of particles left is zero for $\omega_V$ between $0.52$ and $0.7$, between $0.28$ and $0.3$, and there is a lowering in the number of particles left for $\omega_V = 0.2$. 
%add plots with particles left plotted against omega_V

 The behavior of the simulated particles in the time-dependent electric field can be explained with resonance. As the external electric force is a periodic, time-dependent force, it is natural to explain the behavior of the particles with resonance. If a calcium ion had a resonance frequency equal to the frequency $\omega_V$ of the electric field, the oscillation of the particle would be amplified. This is seemingly what is happening in our simulations. As the frequency of the electric field approaches the resonance frequency of calcium ions, 

We can also see that the breadth of the interval of frequencies where the particles all leave the trap increases as the magnitude of the electric field oscillation increases. As the magnitude increases, the oscillations become bigger.

%ref på at 1/n av resonance freq også er resonance frequency
The largest interval where the particles all leave the penning trap corresponds to a resonance frequency of about $0.6$. Assuming that $0.6$ is a resonance frequency, half of that would also be a resonance frequency. For $\omega_V=0.4, 0.7$ we see that particles also are lost out of the trap at a frequency of about $0.3=\dfrac{0.6}{2}$. We would also expect to see that a third of the frequency is another resonance frequency, and in fact we do see a reduction in particles as $0.2=0.6/3$ for $f=0.7$. The behavior of the paricles is in other words consistent with what we would expect from a system with a resonance frequency of about $0.6$.

Further work can be done on resonance within penning trap simulations. It could be interesting to check if there are any resonance frequencies outside of the interval we have investigated. A natural place to start would be in the frequencies below $0.2$. Given that $0.30$ is a resonance frequency, $0.15$, $0.075$, and so on should also be resonance frequencies. It could also be quite useful to study the specific resonance frequency of approximately $0.6$ to better pinpoint the value of the frequency. More simulations of particles, in a smaller interval around $f=0.6$ could give us a better insight. It could also be informative to investigate the resonance frequencies of calcium ions in a real, non-idealized Penning Trap. 
