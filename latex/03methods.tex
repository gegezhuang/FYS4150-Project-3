We model the particles being trapped inside the Penning trap with ordinary differential equations (ODEs) derived from foundational physics equations. Let us quickly go through them here, before using them to argue our model is valid. The first of these equations is Newton's second law (\ref{eq:fun_newton}), which we use to say something about how our particles' velocity is changed in relation to outside forces. Here $m$ is the mass of the particle, while $r$ is its position in Euclidean space. $F_i$ is the $i$th force.
%
\begin{equation}
\label{eq:fun_newton}
m \ddot r = \sum_i F_i
\end{equation}
%
Next comes the equation for how strong an electric field is at a point $r$, when the field is set up by point charges (\ref{eq:fun_couloumb}). In this equation we introduce $k_e$ as Couloumb's constant, $q$ as charge, and say that $q_j$ and $r_j$ are the $j$th particles' charge and position respectively.
%
\begin{equation}
\label{eq:fun_couloumb}
E = k_e \sum_{j=1}^n q_j \frac{r - r_j}{\abs{r - r_j}^3}
\end{equation}
%
Electric fields ($E$) and potential ($V$) are related through (\ref{eq:fun_field_potential}).
\begin{equation}
\label{eq:fun_field_potential}
E = - \nabla V
\end{equation}
%
Lastly, the Lorentz force (\ref{eq:fun_lorentz}) is the force acting on a charged particle moving through an electrical and magnetic field, where the magnetic field is represented by $B$.
%
\begin{equation}
\label{eq:fun_lorentz}
F = qE + qv \times B 
\end{equation}

In addition to these fundamental equations, we have two equations describing how an idealized Penning trap work, that we need to use for our model as well. The electric potential inside the trap is defined by (\ref{eq:potential}), where $V_0$ is the potential applied at the terminals, $d$ is the characteristic distance, a dimensionless size, and $x$, $y$ and $z$ together make up a position.
%
\begin{equation}
\label{eq:potential}
V(x, y, z) = \frac{V_0}{2d^2}(2z^2 - x^2 - y^2)
\end{equation}
%
The magnetic field in the trap is defined as in (\ref{eq:magnetic}). Here, $B_0$ is the field strength.
%
\begin{equation}
\label{eq:magnetic}
B = B_0 \hat{e}_z = (0, 0, B_0)
\end{equation}

With these equations, we can start to model how particles would behave in proximity of or inside the trap. Using (\ref{eq:fun_newton}) and (\ref{eq:fun_lorentz}), and then inserting the properties of our electrical (\ref{eq:potential}) and magnetic (\ref{eq:magnetic}) fields, we get
%
\begin{align*}
m \ddot r &= qE + q\dot r \times B \\
0 &= - m \ddot r + q(- \nabla V) + q\dot r \times B_0 \hat{e}_z \\
0 &= - m \ddot r + q \frac{V_0}{2d^2} \begin{pmatrix} -2x \\ -2y \\ 4z \end{pmatrix}
          + q \begin{pmatrix} \dot y B_0 \\ -\dot x B_0 \\ 0 \end{pmatrix} \\
0 &= \frac{1}{m}
    \begin{pmatrix} 
        \ddot x - q V_0 x / d^2 - q \dot y B_0 \\
        \ddot y - q V_0 y / d^2 + q \dot x B_0 \\
        \ddot z + q V_0 2z / d^2 \\
    \end{pmatrix}
\end{align*}
%
If we now introduce $\omega_0 = \frac{q B_0}{m}$ and $\omega_z^2 = \frac{2qV_0}{md^2}$, and split the vector equations into separate equations, we have that
% Vi burde si noe om tolkningen til omega-variablene, men jeg vet ikke helt hva den er jeg
%
\begin{align}
\label{eq:ode_x} \ddot x - \omega_z^2 x / 2 - \omega_0 \dot y & = 0 \\
\label{eq:ode_y} \ddot y - \omega_z^2 y / 2 + \omega_0 \dot x &= 0 \\
\label{eq:ode_z} \ddot z + \omega_z^2 z &= 0
\end{align}
%
This is a set of three ODEs, where the two first of them are coupled. To solve them more easily, we introduce the complex function $f = x + i y$. Differentiating this twice with regards to time gives us two other equations that we can use to rewrite (\ref{eq:ode_x}) and (\ref{eq:ode_y}) into one complex equation.
%
\begin{equation*}
\dot f = \dot x + i \dot y, \text{ and } \ddot f = \ddot x + i \ddot y.
\end{equation*}
%
We begin with (\ref{eq:ode_x}), and insert expressions for $\ddot x$, $\dot y$ and $x$ derived from our expression for $f$.
%
\begin{align*}
\ddot x - \omega_0 \dot y + \frac{1}{2} \omega_z^2 x &= 0 \\
\left( \ddot f - i \ddot y \right) - \omega_0 \left( \frac{\dot f \dot x}{i} \right) + \frac{1}{2} \omega_z^2 \left( f - iy \right) &= 0
\end{align*}
%
We then insert an expression for $\ddot y$ derived from $f$, and get
%
\begin{align*}
\ddot f - i \left( - \omega_0 \dot x + \frac{1}{2} \omega_z^2 y \right) + i \omega_0 \dot f \dot x + \frac{1}{2} \omega_z^2 \left( f - iy \right) &= 0 \\
\ddot f + i \omega_0 \dot x - \omega_0 \dot x + \frac{1}{2} \omega_z^2 i y - \frac{1}{2} \omega_z^2 i y + i \omega_0 \dot f - \frac{1}{2} \omega_z^2 f &= 0 \\
\ddot f + i \omega_0 \dot f - \frac{1}{2} \omega_z^2 f &= 0
\end{align*}
%
We now have two ODEs that together describe how particles behave in and around our trap. $\ddot f + i \omega_0 \dot f - 1/2 \omega_z^2 f = 0$ gives us information about the $x$-axis ($\Re(f)$) and $y$-axis ($\Im(f)$), and $\ddot z + \omega_z^2 z = 0$ describes the movement along the $z$-axis.

Let us now analyse how our model predicts the particles will behave in the $xy$-plane. The analytical solution of our complex ODE is 
%
\begin{equation}
f(t) = A_+e^{-\omega_+ t} + A_-e^{-\omega_- t},
\label{eq:analytical_solution_xy}
\end{equation}
%
where
\begin{equation*}
\omega_\pm = \frac{\omega_0 \pm \sqrt{\omega_0^2 - 2 \omega_z^2}}{2},
\end{equation*}
and $A_\pm$ are constants set after inserting initial conditions.
%
If we want to trap the particles, we want $\abs{f(t)} < \infty$ even when $t \to \infty$. This means the real component of the exponent must be non-positive, in other words $\text{Re}(-i\omega_\pm) \leq 0$. It could also be that only one of our two terms have non-positive exponents, and $A$ associated with the other term is $0$, but this is highly dependent on initial conditions, and thus very unstable. To understand the relationship between our trap's parameters ($B_0$, $V_0$ and $d$) and the particle's properties ($q$ and $m$), let us expand our $\omega_\pm$-expression and solve the resulting equation.
%
\begin{align*}
0 &\geq \text{Re} \left(-i \omega_\pm \right) \\
  &= \text{Im} \left(- \omega_\pm \right) \\
  &= \frac{1}{2} \text{Im} \left(- \omega_0 \mp \sqrt{\omega_0^2 - 2 \omega_z^2} \right) \\
  &= \text{Im} \left(- \omega_0 \right) + \text{Im} \left( \mp \sqrt{\omega_0^2 - 2 \omega_z^2} \right) \\
  &= \mp \text{Im} \left( \sqrt{\omega_0^2 - 2 \omega_z^2} \right).
\end{align*}
%
This means the imaginary component of the square root must have a strict equality with $0$, which only happens if $\omega_0^2 - 2 \omega_z^2 > 0$, meaning that we get
%
\begin{align*}
\frac{4V_0}{d^2 B_0^2} &< \frac{q}{m},
\end{align*}
%
if we insert our expressions for $\omega_0$ and $\omega_z^2$. This last equation gives us some opportunity to interpret how the particle should behave. The key insight it provides is perhaps that we have to balance the electrical field with that keeps the particles centered in the $z$-axis with the magnetic field that centers it in the $xy$-plane, because the electrical field also exerts an outwards-pushing force on the particle in the $xy$-plane. It also means that the higher the charge of the particle, and lower mass it has, the higher the potential applied to the terminals can be without it pushing the particle out of the trap in the $xy$-plane.

The fact that the real component of the exponents has to be $0$ also lets us know something about the upper and lower bounds for the distance the particles can have from the origin in the $xy$-plane. 

%%% ------------------------------------------------------------------------ %%%
% Har bare ført inn en introduksjon til problem 4, men tror problem 1-3 er helt 
% ok ført nå. I problem 4 husker jeg ikke og klarer ikke finne en forklaring på
% del 2, men del 1 har jeg kontroll på.
% Det er litt mange ligninger og mye matte per forklaring, men det er mulig det 
% er vanskelig å unngå. Kan hende vi kan slippe unna med mindre mellomregning 
% da.
%%% ------------------------------------------------------------------------ %%%
