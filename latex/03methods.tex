We model the particles being trapped inside the Penning trap with ordinary differential equations (ODEs) derived from foundational physics equations. Let us quickly go through them here, before using them to argue our model is valid. The first of these equations is Newton's second law (\ref{eq:fun_newton}), which we use to say something about how our particles' velocity is changed in relation to outside forces.
%
\begin{equation}
\label{eq:fun_newton}
m \ddot r = \sum_i F_i
\end{equation}
%
Next comes Coloumb's law (\ref{eq:fun_coloumb}), that describes how electrically charged particles interact. Here, $k_e$ is Couloumb's constant.
%
\begin{equation}
\label{eq:fun_coloumb}
E = k_e \sum_{j=1}n q_j \frac{r - r_j}{\abs{r - r_j}^3}
\end{equation}
%
Electric fields and potential are related through (\ref{eq:fun_field_potential}).
\begin{equation}
\label{eq:fun_field_potential}
E = - \nabla V
\end{equation}
%
Lastly, the Lorentz force (\ref{eq:fun_lorentz}) is the force acting on a charged particle moving through an electrical and magnetic field.
%
\begin{equation}
\label{eq:fun_lorentz}
F = qE + qv \times B 
\end{equation}

%%% ^^^^^^^^^^^^^^^^^^^ DISKUSJON OM AVSNITTENE OVER HER ^^^^^^^^^^^^^^^^^^^ %%%
% Burde vi her gå i større detalj og forklare hva alle variablene er (Didrik lener mot ja)?
% Og burde vi droppe hele denne greien? Eller eventuelt flytte den til en appendiks?
%%% ------------------------------------------------------------------------ %%%

In addition to these fundamental equations, we have two equations describing how an idealized Penning trap work, that we need to use for our model as well. The electric potential inside the trap is defined by (\ref{eq:potential}), where $V_0$ is the potential applied at the terminals, and $d$ is the characteristic distance, a dimensionless size.
%
\begin{equation}
\label{eq:potential}
V(x, y, z) = \frac{V_0}{2d^2}(2z^2 - x^2 - y^2)
\end{equation}
%
The magnetic field in the trap is defined as in (\ref{eq:magnetic}).
%
\begin{equation}
\label{eq:magnetic}
B = B_0 \hat{e}_z = (0, 0, B_0)
\end{equation}

With these equations, we can start to model how particles would behave in proximity of or inside the trap. Using (\ref{eq:fun_newton}) and (\ref{eq:fun_lorentz}), and then inserting the properties of our electrical (\ref{eq:potential}) and magnetic (\ref{eq:magnetic}) fields, we get
%
\begin{align*}
m \ddot r &= qE + q\dot r \times B \\
0 &= - m \ddot r + q(- \nabla V) + q\dot r \times B_0 \hat{e}_z \\
0 &= - m \ddot r + q \frac{V_0}{2d^2} \begin{pmatrix} -2x \\ -2y \\ 4z \end{pmatrix}
          + q \begin{pmatrix} \dot y B_0 \\ -\dot x B_0 \\ 0 \end{pmatrix} \\
0 &= \frac{1}{m}
    \begin{pmatrix} 
        \ddot x - q \frac{V_0}{d^2} x - q \dot y B_0 \\
        \ddot y - q \frac{V_0}{d^2} y + q \dot x B_0 \\
        \ddot z + q \frac{V_0}{d^2} 2z \\
    \end{pmatrix}
\end{align*}
%
If we now introduce $\omega_0 = \frac{q B_0}{m}$ and $\omega_z^2 = \frac{2qV_0}{md^2}$, we can write the previous equation as
%
\begin{align*}
0 &= \frac{1}{m}
    \begin{pmatrix} 
        \ddot x - \frac{1}{2} \omega_z^2 x - \omega_0 \dot y \\
        \ddot y - \frac{1}{2} \omega_z^2 y + \omega_0 \dot x \\
        \ddot z + \omega_z^2 z \\
    \end{pmatrix}
\end{align*}
%
This is a set of three ODEs, where the two first of them are coupled. To solve them more easily, we introduce the complex function $f(t) = x(t) + i y(t)$. Differentiating this twice with regards to time gives us two other equations that we can use to rewrite rewrite the first and second component of our vector equation into one complex equation.
%
\begin{align}
\dot f(t) &= \dot x(t) + i \dot y(t) \\
\ddot f(t) &= \ddot x(t) + i \ddot y(t),
\end{align}
%
We begin with our first component, and insert expressions for $\ddot x$, $\dot y$ and $x$ derived from our expression for $f$.
%
\begin{align*}
\ddot x - \omega_0 \dot y + \frac{1}{2} \omega_z^2 x &= 0 \\
\left( \ddot f - i \ddot y \right) - \omega_0 \left( \frac{\dot f \dot x}{i} \right) + \frac{1}{2} \omega_z^2 \left( f - iy \right) &= 0 \\
\end{align*}
%
We then insert an expression for $\ddot y$ derived from $f$, and get
%
\begin{align*}
\ddot f - i \left( - \omega_0 \dot x + \frac{1}{2} \omega_z^2 y \right) + i \omega_0 \dot f \dot x + \frac{1}{2} \omega_z^2 \left( f - iy \right) &= 0 \\
\ddot f + i \omega_0 \dot x - \omega_0 \dot x + \frac{1}{2} \omega_z^2 i y - \frac{1}{2} \omega_z^2 i y + i \omega_0 \dot f - \frac{1}{2} \omega_z^2 f &= 0 \\
\ddot f + i \omega_0 \dot f - \frac{1}{2} \omega_z^2 f &= 0
\end{align*}
%
We now have two ODEs that together describe how particles behave in and around our trap. $\ddot f + i \omega_0 \dot f - 1/2 \omega_z^2 f = 0$ gives us information about the $x$-axis ($\Re(f)$) and $y$-axis ($\Im(f)$), and $\ddot z + \omega_z^2 z = 0$ describes the movement along the $z$-axis. 

%%% ------------------------------------------------------------------------ %%%
% Har ikke ført inn problem 3, men tror problem 1 og 2 er helt ok ført nå. Det er
% litt mange ligninger og mye matte per forklaring, men det er mulig det er 
% vanskelig å unngå. Kan hende vi kan slippe unna med mindre mellomregning da,
% og kanskje å droppe de ligningene helt øverst.
%%% ------------------------------------------------------------------------ %%%

\section*{Oppgave 3}
$\text{Re}(-i\omega_\pm) \leq 0$
