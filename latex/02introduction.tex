In this report, we are examining how particles behave in a penning trap. A penning trap is a device that applies a homogenous magnetic field and an inhomogenous electric field in an area to trap a charged particle. The penning trap can be a very useful tool for physicists, which is why we are using numerical simulations to study it.
\\\\
In particular we will look at $\text{Ca}^+$-ions in a penning trap with a time-dependent, periodic electric potential. When applying a time-dependent force to particles, such as the force from our electric field, resonance can occur. We intend to use numerical simulations to find out under what circumstances resonance occurs in the penning trap. More concretely, we will use the algorithm Runge-Kutta 4 to solve the equations of motion of the particles numerically.
\\\\
In the methods-section we will first explain the physics we use in the report. Next, we derive an analytical solution for the movement of a single particle in a penning trap with a constant electric potential, and bounds for solutions where the particles remain in the trap. We wrap up the methods-section by explaining the algorithm Runge-Kutta 4. In the results and discussion-section we first look at the performance of Runge-Kutta 4. Next, we find resonance frequencies of $\text{Ca}^+$-ions by simulating a penning trap with a time-dependent electric potential. We proceed to simulate 100 particles in the penning trap for a resonance frequency, both with and without coulomb interactions between particles. 
